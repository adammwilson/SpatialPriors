a. Illustration of a spatial Prior Relative Occurence Rate (PROR).  Thin black lines represent a subset of the 'expert' range map for the Montane Woodcreeper (\textit{Lepidocolaptes lacrymiger}) and the points are occurrence observations extracted from eBird.  Colors indicate the normalized logistic distance-decay (prob=0.54, rate=0.1, skew=0.2) with the upper and lower values estimated to achieve 54\% probability inside the expert range. Thick line is the transect illustrated in panel B.  b. Illustration of various decay curves across transect shown in panel a showing the effects of varying the rate ($r$), skew ($S$), and $P_{in}$. X-axis has an inverse hyperbolic sine transformation and y-axis is log-transformed.   c. Illustration of feasible decay parameters given the range and domain geometry for this species.  Colors indicate the difference between the desired $P_{in}$ and the maximum possible  $P_{in}$ given the specified curve.  Contour lines show the distance buffer needed around the expert range to include the desired  $P_{in}$.
  