\section{Making binary maps}%----------------------


An alternative way to assign probabilities to the expert map is also based on omission rates, but avoids the assumption that the true ommision rate is known. However, the tradeoff is that one can only make a binary prediction, analogous to the expert map. This approach relies on the interpretation of Maxent's 'cumulative' output format. A cell's cumulative output value is equal to 100 times the sum of raw values of all cells with a smaller raw value than that cell. Cumulative output is useful for creating binary maps; choosing a prediction threshold at a cumulative value of (e.g.) 10 implies that one is constructing a binary map that is expected to omit 10\% of presences (Merow et al. 2013). If one chooses the prediction threshold to equal the expert map's ommission rate, the choice of this threshold effectively cancels out from the binary prediction. That is, the predicted binary map is an updated version of the expert map with the same expected ommission rate (whatever that happens to be). 
